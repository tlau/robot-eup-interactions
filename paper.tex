\documentclass{article}
\begin{document}

\title {Why robot programming is hard}
\author {Tessa Lau \\
Willow Garage \\
Menlo Park, CA USA \\
{\tt tlau@willowgarage.com}
}
\maketitle

Why robotics is becoming important
Statistics on size of market
Number of research investments?
Roomba, Neato in consumer use
	http://www.neatorobotics.com/
	http://store.irobot.com/home/index.jsp
Baxter human-safe 2-armed manipulator
	runs ROS
	http://www.ros.org/news/2012/09/rethink-ros.html
	http://www.rethinkrobotics.com/index.php/products/baxter/
Industrial robots:
	KUKA robot arm http://en.wikipedia.org/wiki/KUKA
Toyota's human support robot:
	runs ROS
	http://www.gizmag.com/toyota-human-support-robot/24246/
Telepresence robots:
	Beam
	Anybot QB
	Double

Main points:
	EUP is what will enable robotics to spawn new industries
	by enabling businesses to customize robot behavior for their own needs

	Yet programming robots is difficult
	State of the art today is demonstrational, which works well for limited, repetitive tasks
	Or using highly technical tools to specify fine-grained robot behavior
	Behavior specification requires much deep technical knowledge
	Not only the traditionally hard topics in EUP such as conditionals, loops, variables
	But also advanced topics like concurrency, uncertainty, high-dimensionality

\section{ROS}

History of ROS
Invented by Willow Garage, new operating system for diverse robots
Current reach
running on robots from PR2 to android-powered XXX
Provides very rich control substrate for programming the behavior of complex robots

\section{Concurrency}

ROS nodes
multiple parts of robots (head, arms, base)
callbacks, coroutines, threads

\section{Uncertainty}

World changes out from under you
Hardware fails
Incomplete world models
Bayesian methods, probabilities

\section{High-dimensionality}

Robot arms have 7DOF
grasps require 6D
world is 3D
quaternions, matrix manipulation
specialized I/O devices

\bibliographystyle{plain}
\bibliography{general}

\end{document}

